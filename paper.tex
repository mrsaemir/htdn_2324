\documentclass[10pt,sigconf]{acmart}
\settopmatter{printacmref=false} % Removes citation information below abstract
\renewcommand\footnotetextcopyrightpermission[1]{} % removes footnote with conference information in first column
\pagestyle{plain}
\usepackage[english]{babel}
\usepackage{blindtext}



\begin{document}
\title{A \LaTeX\ Template for HTDN 2020}

\subtitle{Fancy subtitle}
 \author{Amirhossein Saemi}
% \authornote{Note}
% \orcid{1234-5678-9012}
% \affiliation{%
%   \institution{Affiliation}
%   \streetaddress{Address}
%   \city{City} 
%   \state{State} 
%   \postcode{Zipcode}
% }
 \email{amsa00004@uni-saarland.de}


\begin{abstract}
In a time where television and radio no longer dominate as the sole forms of mass communication, the Internet emerges as a significant potential adversary to the spread of propaganda. Consequently, world governments often strive to develop technologies that allow them to exert control over the internet and maintain their monopoly as information empires. 

While censorship may not be a new concept, it remains an effective tool in combating the spread of undesirable information. This paper examines the development of Internet censorship and how users have countered it by employing different circumvention tools. A specific focus is placed on the examination of the Great Firewall of China and how it detects fully encrypted traffic in real-time. Despite the GFW's blackbox nature, we explore how supporters of a free Internet have developed novel escape strategies based on GFW's observed behavior when probing with specially crafted payloads.

Next, we explore how and why world governments make use of a variety of techniques in parallel to exert greater control over the Internet. Ultimately, we conclude our review by advocating for the automation of censorship circumvention through collaborative machine analysis and artificial intelligence.
\end{abstract}

\maketitle

\section{Introduction}
\blindtext

\section{Related Work}
\blindtext

\section{Background}

\subsection{Evolution of Censorship Techniques}
\blindtext

\subsection{From Centralized to Distributed Censorship}
\blindtext

\subsection{Stateful Censorship}
\blindtext

\subsection{Circumvention Technologies}
\blindtext

\section{Mechanics of Real-time Censorship in GFW}
\blindtext

\subsection{Mismatching Fingerprints}
\blindtext

\subsection{Flow Reassembly}
\blindtext

\subsection{How the GFW Disrupts a Connection}
\blindtext

\subsection{Residual Censorship}
\blindtext

\section{Evading Real-time Censorship}
\blindtext

\subsection{Detecting Exemption Rules}
\blindtext

\subsection{Discovering Rules}
\blindtext

\subsection{Evaluating Discovered Rules}
\blindtext

\subsection{Circumvention Strategies}
\blindtext

\section{Combining Multiple Censoring Techniques}
\blindtext

\subsection{The Role of Active Prober}
\blindtext

\section{Locating Network Middleboxes}
\blindtext

\section{Conclusion}
\blindtext

\section{Summary}
\blindtext

\bibliographystyle{ACM-Reference-Format}
\bibliography{reference}

\end{document}
