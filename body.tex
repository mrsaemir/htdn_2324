\section{Introduction}

\section{Related Work}

\section{Background}

\subsection{Evolution of Censorship Techniques}
In parallel with the continuous expansion of internet users, diverse web services and social media apps that have become integral to modern life, censorship techniques are also actively evolving by world governments. 

As a response, the anti-censorship community is committed to developing and distributing circumvention tools for accessing the unrestricted Internet. Consequently, governments have shifted from simple and cost-effective approaches to more complex, expensive, and harder-to-circumvent censorship techniques.

These methods vary from simple DNS poisoning to advanced technologies like Deep Packet Inspection (DPI), allowing them to make decisions on a per-connection basis. In extreme cases, certain countries are even willing to pay astronomical prices, resorting to complete internet shutdowns.

Some countries have taken further steps by developing tools to weaponize innocent users, turning them into botnets used for attacking and disrupting popular VPN service providers. China's Great Canyon is a specific example, where malicious JavaScript payloads are sent to users to perform DDoS attacks on targeted websites.\cite{marczak2015great}

\subsection{From Centralized to Distributed Censorship}
Given the decentralized structure of the Internet, it is impractical to route the entire traffic of a country through a single device or a cluster of censoring devices. Instead, each Internet Service Provider (ISP) is responsible for implementing censorship individually, ideally at the network edge. The management of these middleboxes can be established either centrally, directly by authorities, or individually by each ISP, depending on their respective policies.

In some countries, like Russia, a centralized control model is implemented, where the authority maintains control over a network of numerous Internet Service Providers (ISPs). This centralized approach grants substantial power to the authorities, allowing them to unilaterally impose desired restrictions. In contrast, countries such as Iran follow a distributed model, resulting in potential inconsistencies in censorship behavior among different ISPs, leading to varied experiences for users.\cite{xue2021throttling}

In India, the authorities shared that while the censorship policies are confidential, the onus of implementing them lies with the individual ASes who could employ any mechanism they choose.\cite{yadav2018light}

While a centralized approach may offer a faster response and the ability to swiftly implement blocks, distributed control could prove more resilient in combating constantly updated circumvention tools. This is because circumvention tools must take into account the diverse policies and technologies employed by different ISPs in a distributed model.

\subsection{Stateful Censorship}

\subsection{Circumvention Technologies}

\section{Mechanics of Real-time Censorship in GFW}

\subsection{Mismatching Fingerprints}

\subsection{Flow Reassembly}

\subsection{How the GFW Disrupts a Connection}

\subsection{Residual Censorship}

\section{Evading Real-time Censorship}

\subsection{Detecting Exemption Rules}

\subsection{Discovering Rules}

\subsection{Evaluating Discovered Rules}

\subsection{Circumvention Strategies}

\section{Combining Multiple Censoring Techniques}

\subsection{The Role of Active Prober}

\section{Locating Network Middleboxes}

\section{Conclusion}

\section{Summary}